%%%%%%%%%%%%%%%%%
% This is an example CV created using altacv.cls (v1.1.5, 1 December 2018) written by
% LianTze Lim (liantze@gmail.com), based on the
% Cv created by BusinessInsider at http://www.businessinsider.my/a-sample-resume-for-marissa-mayer-2016-7/?r=US&IR=T
%
%% It may be distributed and/or modified under the
%% conditions of the LaTeX Project Public License, either version 1.3
%% of this license or (at your option) any later version.
%% The latest version of this license is in
%%    http://www.latex-project.org/lppl.txt
%% and version 1.3 or later is part of all distributions of LaTeX
%% version 2003/12/01 or later.
%%%%%%%%%%%%%%%%

%% If you are using \orcid or academicons
%% icons, make sure you have the academicons
%% option here, and compile with XeLaTeX
%% or LuaLaTeX.
% \documentclass[10pt,a4paper,academicons]{altacv}

%% Use the "normalphoto" option if you want a normal photo instead of cropped to a circle
% \documentclass[10pt,a4paper,normalphoto]{altacv}

\documentclass[10pt,a4paper,ragged2e]{altacv}

%% AltaCV uses the fontawesome and academicon fonts
%% and packages.
%% See texdoc.net/pkg/fontawecome and http://texdoc.net/pkg/academicons for full list of symbols. You MUST compile with XeLaTeX or LuaLaTeX if you want to use academicons.

% Change the page layout if you need to
\geometry{left=2cm,right=2cm,marginparwidth=6.8cm,marginparsep=1.2cm,top=1.25cm,bottom=1.25cm}

% Change the font if you want to, depending on whether
% you're using pdflatex or xelatex/lualatex
\ifxetexorluatex
  % If using xelatex or lualatex:
  \setmainfont{Carlito}
\else
  % If using pdflatex:
  \usepackage[utf8]{inputenc}
  \usepackage[T1]{fontenc}
  \usepackage[default]{lato}
\fi

% Change the colours if you want to
\definecolor{VividPurple}{HTML}{000000}
\definecolor{SlateGrey}{HTML}{2E2E2E}
\definecolor{LightGrey}{HTML}{2E2E2E}
\colorlet{heading}{VividPurple}
\colorlet{accent}{VividPurple}
\colorlet{emphasis}{SlateGrey}
\colorlet{body}{LightGrey}

% Change the bullets for itemize and rating marker
% for \cvskill if you want to
\renewcommand{\itemmarker}{{\small\textbullet}}
\renewcommand{\ratingmarker}{\faCircle}

%% sample.bib contains your publications
\addbibresource{sample.bib}

% MANUAL USEPACKAGES
\usepackage{multicol}

\begin{document}
\name{J\'ulia Begrmann}
\tagline{Data Scientist \& Musician}
% Cropped to square from https://en.wikipedia.org/wiki/Marissa_Mayer#/media/File:Marissa_Mayer_May_2014_(cropped).jpg, CC-BY 2.0
%\photo{3.3cm}{profile.jpg}
\personalinfo{%
	% Not all of these are required!
	% You can add your own with \printinfo{symbol}{detail}
	\email{julia@oszterberg.com}
	%   \phone{000-00-0000}
	%  \mailaddress{Address, Street, 00000 County}
	\location{Budapest, Hungary}
	%  \homepage{robertonoel.com/}
	%  \twitter{@marissamayer}
	\linkedin{linkedin.com/in/juliabrgmnn/}
	\github{github.com/juliabergmann} % I'm just making this up though.
	%   \orcid{orcid.org/0000-0000-0000-0000} % Obviously making this up too. If you want to use this field (and also other academicons symbols), add "academicons" option to \documentclass{altacv}
}

%% Make the header extend all the way to the right, if you want.
\begin{fullwidth}
	\makecvheader
\end{fullwidth}

%% Depending on your tastes, you may want to make fonts of itemize environments slightly smaller
\AtBeginEnvironment{itemize}{\small}

%% Provide the file name containing the sidebar contents as an optional parameter to \cvsection.
%% You can always just use \marginpar{...} if you do
%% not need to align the top of the contents to any
%% \cvsection title in the "main" bar.
% \divider
%\cvskill{German}{3}

\begin{multicols}{2}

	% EDUCATION
	\cvsection{Education}
	\cvevent{E\"otv\"os Lor\'and University}{PhD in Informatics}{Sep 2020 -- Jun 2025}{Budapest}

	\begin{itemize}
		\item Industrial data analytics\\
		\item Defining and refining production system models\\
		\item Machine learning
	\end{itemize}

	\divider

	\cvevent{Budapest University of Technology}{MSc in Mathematics}{Sep 2015 -- Jun 2018}{Budapest}
	\begin{itemize}
		\item Specialized on Analysis
	\end{itemize}

	\divider

	\cvevent{Budapest University of Technology}{BSc in Mathematics}{Sep 2015 -- Jun 2018}{Budapest}

	% EXPERIENCE
	\cvsection{Experience}
	\cvevent{Data Analyst \& Scientist}{EPIC InnoLabs}{Sep 2018 -- Present}{Budapest}
	\begin{itemize}
		\item blah\\
		\item blah\\
		\item blah
	\end{itemize}

	\divider

	\cvevent{Researcher}{SZTAKI}{Oct 2017 -- Present}{Budapest}
	\begin{itemize}
		\item blah\\
		\item blah
	\end{itemize}

	\divider

	\cvevent{Trainee}{Nokia}{Sep 2016 -- Jan 2017}{Budapest}
	\begin{itemize}
		\item blah
	\end{itemize}


	\cvsection{Languages}
	\cvskill{English}{5}
	\cvskill{French}{3}
	\cvskill{Spanish}{1}
	% \cvevent{Product Engineer}{Google}{23 June 1999 -- 2001}{Palo Alto, CA}

	% \begin{itemize}
	% \item Joined the company as employe \#20 and female employee \#1
	% \item Developed targeted advertisement in order to use user's search queries and show them related ads
	% \end{itemize}

	%\cvsection{A Day of My Life}

	% Adapted from @Jake's answer from http://tex.stackexchange.com/a/82729/226
	% \wheelchart{outer radius}{inner radius}{
	% comma-separated list of value/text width/color/detail}
	% Some ad-hoc tweaking to adjust the labels so that they don't overlap
	% \wheelchart{1.5cm}{0.5cm}{%
	%   10/10em/accent!30/Sleeping \& dreaming about work,
	%   25/9em/accent!60/Public resolving issues with Yahoo!\ investors,
	%   5/13em/accent!10/\footnotesize\\[1ex]New York \& San Francisco Ballet Jawbone board member,
	%   20/15em/accent!40/Spending time with family,
	%   5/8em/accent!20/\footnotesize Business development for Yahoo!\ after the Verizon acquisition,
	%   30/9em/accent/Showing Yahoo!\ employees that their work has meaning,
	%   5/8em/accent!20/Baking cupcakes
	% }
	\cvsection{TECHNICAL SKILLS}
	\cvtag{Python}
	\cvtag{SQL}
	\cvtag{Plotly}
	\cvtag{Dash}
	\cvtag{NumPy}
	\cvtag{Pandas}
	\cvtag{Scikit-learn}
	\cvtag{Tensorflow}
	\cvtag{Pycaret}
	\cvtag{R}
	\cvtag{Shiny}
	\cvtag{Plant Simulation}
	\cvtag{Xpress}
	\cvtag{\LaTeX}
	%\divider
	% \cvachievement{\faTrophy}{}{Received accolades at Atos for Best Performance in team.}
	% \cvachievement{\faTrophy}{}{Received Best Debut Award at Atos. }
	% %\divider
	% \cvachievement{\faInstitution}{}{Won 2nd Consolation Prize for paper presented on Cognitive Radio Networks.}
	% %\divider
	% \cvachievement{\faGraduationCap}{}{Got Selected in "Exclusive Scholar Program" during undergrad.}
	% %\divider
	% \cvachievement{\faDollar}{}{Awarded with Narotam Sekhsaria Foundation Scholarship}
	%\cvsection{Strengths}

	%\cvtag{Hard-working (18/24)} 
	%\cvtag{Persuasive}
	%\cvtag{Motivator \& Leader}

	%\divider\smallskip

	%\cvtag{UX}
	%\cvtag{Mobile Devices \& Applications}
	%\cvtag{Product Management \& Marketing}


	%\divider

	%\cvevent{B.S.\ in Symbolic Systems}{Stanford University}{Sept 1993 -- June 1997}{}

	\cvsection{Projects}
	\cvproject{Project}
	\begin{itemize}
		\item Synthesizing Music Information Retrieval (MIR) and genetic algorithms to discover and improve short, music generating lines of C code.
		\item Implementing the model using a similar approach to the Ramanujan Machine.
	\end{itemize}
	\smallskip
	\cvproject{Project}
	\begin{itemize}
		\item Designing and developing an electronic health records system focusing on aiding and assisting patients as well as providing a smoother data management experience for healthcare personnel (e.g. Physicians, Nurses, etc.).
	\end{itemize}
	\smallskip
	\cvproject{Project}
	\begin{itemize}
		\item Developed an MIR algorithm to detect pitches in .mp3 files, reproduce them in MIDI, and synthesize them using custom wave formats.
		\item Used real-time signal processing to visualize the outputs generated by the algorithm.
	\end{itemize}
	\smallskip
	\cvproject{Project}
	\begin{itemize}
		\item Designed and developed a flight booking website from scratch, providing multiple services to both end-consumers and airlines.
		\item Implemented the website using Python, Flask, SQL, JS and HTML/Bootstrap/CSS.
	\end{itemize}
	\smallskip
	\cvproject{Project}
	\begin{itemize}
		\item Solely developed a program to automatically generate quarterly performance reports for Arch Global Advisors, scraping the web to get historical stock price data, dividend data, and earnings release dates.
		\item Ultimately cut the firm's reporting process from days to minutes.
	\end{itemize}

	\cvsection{Soft Skills}
	\begin{itemize}
		\item Natural networker, who, having lived in several countries, quickly adapts to social environments with people from all over the world.
		\item Hosted Chinese Language Competition at NYU, acting as an MC for an audience of more than 100 attendees.
		\item Acted as a Community Manager for Red Pulse, successfully leading a group of over 150 writers.

	\end{itemize}

	\cvproject{}


\end{multicols}
\clearpage

\cvsection{PUBLICATIONS}


% \cvitem{
%     \cvheadingstyle{}
% }{
\begin{enumerate}[label={[\arabic*]}]
	\item Bergmann, J., Zeleny, K. É., Váncza, J., Kő, A. (2022). Tool failure recognition using inconsistent data. \textit{Procedia CIRP 107}, 1204-1209. \url{https://doi.org/10.1016/j.procir.2022.05.132}
	\item Bergmann, J., Gyulai, D., \& Váncza, J. (2021). Adaptive AGV fleet management in a dynamically changing production environment. \textit{PROCEDIA MANUFACTURING, 54}, 148–153. \url{http://doi.org/10.1016/j.promfg.2021.07.046}
	\item Bergmann, J., Gyulai, D., \& Váncza, J. (2021). Automated vehicle fleet management in manufacturing environments combining network analysis, parameter prediction and optimization techniques. In \textit{IFORS 2021 Virtual, Programme Book} (pp. 128–128). \url{https://www.euro-online.org/conf/admin/tmp/program-ifors2021.pdf}
	\item Frye, M., Gyulai, D., Bergmann, J., \& Schmitt, R. H. (2021). Production rescheduling through product quality prediction. \textit{PROCEDIA MANUFACTURING, 54}, 142–147. \url{http://doi.org/10.1016/j.promfg.2021.07.022}
	\item Bergmann, J., Gyulai, D., Morassi, D., \& Váncza, J. (2020). A stochastic approach to calculate assembly cycle times based on spatial shop-floor data stream. \textit{PROCEDIA CIRP, 93}, 1164–1169. \url{http://doi.org/10.1016/j.procir.2020.03.052}
	\item Gyulai, D., Bergmann, J., Lengyel, A., Kádár, B., \& Czirkó, D. (2020). Simulation-based Digital Twin of a Complex Shop-Floor Logistics System. In \textit{Proceedings of the 2020 Winter Simulation Conference, WSC 2020} (pp. 1849–1860). \url{http://doi.org/10.1109/WSC48552.2020.9383936}
	\item Gyulai, D., Bergmann, J., \& Váncza, J. (2020). Adaptive network analytics for managing complex shop-floor logistics systems. \textit{CIRP ANNALS-MANUFACTURING TECHNOLOGY, 69}(1), 393–396. \url{http://doi.org/10.1016/j.cirp.2020.04.002}
	\item Gyulai, D., Pfeiffer, A., \& Bergmann, J. (2020). Analysis of asset location data to support decisions in production management and control. \textit{PROCEDIA CIRP, 88}, 197–202. http://doi.org/10.1016/j.procir.2020.05.035
	\item Molontay, R., Horváth, N., Bergmann, J., Szekrényes, D. L., \& Szabó, M. (2020). Characterizing curriculum prerequisite networks by a student flow approach. \textit{IEEE TRANSACTIONS ON LEARNING TECHNOLOGIES, 13}(3), 491–501. \url{http://doi.org/10.1109/TLT.2020.2981331}
	\item Tsutsumi, D., Gyulai, D., Takács, E., Bergmann, J., Nonaka, Y., \& Fujita, K. (2020). Personalized work instruction system for revitalizing human-machine interaction. \textit{PROCEDIA CIRP, 93}, 1145–1150. \url{http://doi.org/10.1016/j.procir.2020.04.062}
	\item Frye, M., Gyulai, D., Bergmann, J., \& Schmitt, R. H. (2019). Adaptive scheduling through machine learning-based process parameter prediction. \textit{MM Science Journal, 2019}(November), 3060–3066. \url{http://doi.org/10.17973/MMSJ.2019\_11\_2019051}
	\item Gyulai, D., Bergmann, J., Gallina, V., \& Gaal, A. (2019). Towards a connected factory: Shop-floor data analytics in cyber-physical environments. \textit{PROCEDIA CIRP, 86}, 37–42. \url{http://doi.org/10.1016/j.procir.2020.01.016}
	\item Nick, G., Szaller, Á., Bergmann, J., \& Várgedő, T. (2019). Industry 4.0 readiness in Hungary: model, and the first results in connection to data application. \textit{IFAC PAPERSONLINE, 52}(13), 289–294. \url{http://doi.org/10.1016/j.ifacol.2019.11.185}
	\item Gyulai, D., Pfeiffer, A., Bergmann, J., \& Gallina, V. (2018). Online lead time prediction supporting situation-aware production control. \textit{PROCEDIA CIRP, 78}, 190–195. \url{http://doi.org/10.1016/j.procir.2018.09.071}
\end{enumerate}

% \cvsection[page2sidebar]{Publications}

\nocite{*}

% \printbibliography[heading=pubtype,title={\printinfo{\faBook}{Books}},type=book]

% \divider

% \printbibliography[heading=pubtype,title={\printinfo{\faFileTextO}{Journal Articles}}, type=article]

% \divider

% \printbibliography[heading=pubtype,title={\printinfo{\faGroup}{Conference Proceedings}},type=inproceedings]

% %% If the NEXT page doesn't start with a \cvsection but you'd
% %% still like to add a sidebar, then use this command on THIS
% %% page to add it. The optional argument lets you pull up the
% %% sidebar a bit so that it looks aligned with the top of the
% %% main column.
% % \addnextpagesidebar[-1ex]{page3sidebar}


\end{document}
